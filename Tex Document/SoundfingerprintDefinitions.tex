\section{Sound fingerprint}
Sound fingerprints are the base for .DTSTF files. It contains the result of a wave analyse which will be defined later on. As a result of that it contains neither the analytical calculations nor a reconstruction definition.

\subsection{Contained information}
A sound fingerprint contains several informations. First of all the number of required sine waves what defines how many sine waves will be needed to create an appropriate note. In second place the wave definitions including a frequency and amplitude factor for each sine wave. And last of all the decay functions which will be defined by a function type and a decay factor.

\subsection{Data composition}
\textbf{\emph{Number of sine waves:}}
\begin{quote}
	This value will be defined by three numbers which declare how many sine waves -- with priority high, medium and low -- are required to create an appropriate note.
\end{quote}
\textbf{\emph{Frequency factor:}}
\begin{quote}
	The frequency factor $ f $ defines a value to modify a predefined base frequency $ \mathcal{F} $.
\end{quote}
\begin{eqnarray}
	& sin(f*\mathcal{F}) \quad f\in\N, \; \mathcal{F}\in\R_{>0}\nonumber \\
	(\ref{ToneRef})\Rightarrow & \; T(x)=\mathcal{A} * \gamma(x) * sin(f*\mathcal{F}) \quad x\in\R\nonumber
\end{eqnarray}
\textbf{\emph{Amplitude and decay function}}
\begin{quote}
	These values correspond to the definitions (\ref{AmplitudeRef}) and (\ref{DecayRef}).
\end{quote}

\subsection{File outlining}
\begin{equation}\label{DTSTFDataRef}
	\begin{tabular}{|c|c||d|}
		\hline 
		Section & Priority & Data\\
		\hline\hline
		\multirow{3}{*}{Number of required sine waves} & High & 4\\
		& Medium & 3\\
		& Low & 6\\
		\hline\hline
		Section & Frequency & Amplitude\\
		\hline\hline
		\multirow{13}{*}{Frequency and amplitude factor} 
		& 2 & 0,8\\
		& 4 & 0,6\\
		& 5 & 0,7\\
		& 8 & 0,6\\
		\cline{2-3}
		& 1 & 0,55\\
		& 3 & 0,5\\
		& 9 & 0,4\\
		\cline{2-3}
		& 6 & 0,3\\
		& 7 & 0,4\\
		& 10 & 0,35\\
		& 11 & 0,2\\
		& 12 & 0,12\\
		& 13 & 0,24\\
		\hline\hline
		Section & Type & Value\\
		\hline\hline
		\multirow{13}{*}{Decay function type and value}&
		\multirow{13}{*}{Exponential}
		& 1.3\\
		&& 1.5\\
		&& 1.8\\
		&& 1.1\\
		\cline{3-3}
		&& 1.9\\
		&& 1.6\\
		&& 2.2\\
		\cline{3-3}
		&& 1.9\\
		&& 1.0\\
		&& 2.1\\
		&& 2.6\\
		&& 1.4\\
		&& 1.7\\
		\hline
	\end{tabular}
\end{equation}
\begin{center}
	The data column shows information from the DTSTF format. The other two columns are for description.
\end{center}